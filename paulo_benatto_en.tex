% LaTeX file for resume 
% This file uses the resume document class (res.cls)

\documentclass[margin]{res}
\usepackage [brazil]{babel}     % nomes e hifenaçã em português
 
\usepackage{t1enc}              % Permite digitar os acentos de forma normal
\usepackage[utf8]{inputenc} 

\topmargin=-0.5in  % start text higher on the page
\setlength{\textheight}{10in} % increase text height to fit resume on 1 page
\begin{document}  
\name{\textit{Paulo Leonardo Benatto}}

\address{London, UK \\ benatto@gmail.com \\ Phone: 07405110040 \\ Post Code: SW16 2BU }
                           
                        
\begin{resume}                        
 
\section{Summary}       Bachelor in computer science having expertise in software development, good professional relationship and focused at work.
                        With 6 years of professional experience in software development using basically, C/C++, I am now looking for new challenges
                        and opportunities that allow me to learn new technologies and work with new people.
 
\section{Education}	Universidade Estadual do Oeste do Parana (UNIOESTE),  BS in Computer Science, December 2007.
  
\section{Experience}

\vspace{-0.1in}
   \begin{tabbing}
   \hspace{2.3in}\= \hspace{1.7in}\= \kill % set up two tab positions
    \textbf{DBA}    \>\>\textbf{Dec 2013 - Feb 2014}\\
    \textit{System Analyst}\\        
    \textbf{Main Tecnologies}: Linux, Python, Raspberry PI and C;
   \end{tabbing}\vspace{-20pt}      % suppress blank line after tabbing
    \vspace{2mm}
    I attended of the project to develop a system of vehicle count on Brazilian highways, basically my
    task was read files with a lot of records and parse all information, store in memory and report this
    information in a human readable. This project was developed using language C ANSI and Linux platform. 
    The second project that I attended was developing a parking meter system using python language and
    Raspberry PI. My main task was do inter-process communication using DBus.

\vspace{-0.1in}
   \begin{tabbing}
   \hspace{2.3in}\= \hspace{1.7in}\= \kill % set up two tab positions
    \textbf{SEC+}    \>\>\textbf{Dec 2012 - Dec 2013}\\
    \textit{System Analyst}\\        
    \textbf{Main Tecnologies}: Linux, Python, Django Framework, JavaScript and C;
   \end{tabbing}\vspace{-20pt}      % suppress blank line after tabbing
    \vspace{2mm}
     Back-end development of web system for intelligent monitoring and management of natural disasters
     using Python and the Django framework. Front-end with Javascript (JQuery, Bootstrap, Google Maps API),
     JSON, HTML5, CSS. Modeling and use of object-relational database and geographic objects with
     PostgreSQL/PostGIS.

   \begin{tabbing}
   \hspace{2.3in}\= \hspace{1.7in}\= \kill % set up two tab positions
    \textbf{Digitro Technology - NDS}    \>\>\textbf{Jan 2012 - Dec 2012}\\
    \textit{System analyst}\\   
    \textbf{Main Tecnologies}: Linux, C ANSI, shellscript, protocols: HTTP, SIP, UDP, TCP;
   \end{tabbing}\vspace{-20pt}      % suppress blank line after tabbing
    \vspace{2mm}
    I worked during one year with great team focused in lawful interception. The most of my time I was
    working with C language, GCC compiler and Linux platform. Basicly my work was create "parses"(programs)
    to analyze the content of network packets, extract all information to store on databases. 

    To ease the work we used the libpcap project. libpcap is a system-independent interface for user-level
    packet capture. libpcap provides a portable framework for low-level network monitoring. Applications
    include network statistics collection, security monitoring, network debugging, etc.

    To create tests on the project we used the CUnit API and to build and execute them, we created a jenkins environment.

    I worked a little bit with java, our team was big, and I would like to learn another world. =)
    
    \vspace{22mm}

   \begin{tabbing}
   \hspace{2.3in}\= \hspace{1.5in}\= \kill % set up two tab positions
    \textbf{Digitro Technology - STE}    \>\>\textbf{Set 2008 - Dec 2011}\\
    \textit{System analyst}\\   
    \textbf{Main Tecnologies}: Linux, VoIP, C/C++, shellscript, protocols: UDP, TCP, SIP;
   \end{tabbing}\vspace{-20pt}      % suppress blank line after tabbing
    \vspace{2mm}
    I worked a little bit with embedded systems developing an IP Phone. I was responsible to cross-compile EFL
    (Enlightenment Foundation Libraries) and develop the IP Phone interface. I had to work with a design team,
    this was awesome.

    I worked in a team to develop a softphone (program for making telephone calls over the Internet) using C/C++
    language and Windows platform. I studied a lot SIP, UDP, TCP and others protocols.

    In some moments I worked with shell, lua and python (scripting). I had to administer Linux systems, 
    Asterisk, OpenSER, Kamailio and others.

   \begin{tabbing}
   \hspace{2.3in}\= \hspace{1.5in}\= \kill % set up two tab positions
    \textbf{Virtual Office}    \>\>\textbf{Jan 2008 - Aug 2008}\\
    \textit{Developer Jr}\\
    \textbf{Main Tecnologies}: Linux, VoIP, SIP, Asterisk and Latex;
   \end{tabbing}\vspace{-20pt}      % suppress blank line after tabbing
    \vspace{2mm}
     This was my first formal job and I learned a lot with great professionals. Here I worked with
     Asterisk and Linux administration. We created an asterisk advanced course (using latex). 

     I worked a little bit with PHP and MySQL, to customize the web page administration of our product.

\section{Skills Base}  \textit{Operating System}:  Linux (Debian, Ubuntu, CentOS and others), Windows NT/XP/Vista/7 and OSX;

			\textit{Network Protocols}: IP, TCP, UDP, HTTP, SIP, FTP, SMTP, SNMP, e outros;
  
			\textit{Progamming Languages}: C, Python, JavaScript, plus some experience with lua / Java / Ruby / PHP / HTML;
  
			\textit{Virtualization}: VirtualBox, VMWare, plus some experience with Xen;

			\textit{Languages}: Fluent in Portuguese, Intermediate in English and Spanish;
 
\section{Open Source Projects}
		In my spare time I develop some open source projects (https://github.com/patito/).
		
		\begin{itemize}
		\vspace{2mm}
		\item \textbf{libpenetra}: The libpenetra was created with the goal of studying the windows binary format known as Portable Executable (PE). With libpenetra you can access all information about PE binaries. (\texttt{https://github.com/patito/libpenetra})\vspace{1mm}	
		\item \textbf{libmalelf}: The libmalelf is an evil library that SHOULD be used for good! It was developed with the intent to assist in the process of infecting binaries and provide a safe way to analyze malwares. (\texttt{https://github.com/SecPlus/libmalelf})\vspace{1mm}
		\item \textbf{malelf}: Malelf is a tool that uses libmalelf to dissect and infect ELF binary. (\texttt{https://github.com/SecPlus/malelf})
		\end{itemize}
 
\section{More Info}
    \begin{itemize}
     \item \textbf{Linkedin}: http://www.linkedin.com/in/benatto
     \item \textbf{Github}: https://github.com/patito
    \end{itemize}


\end{resume} 
\end{document}









