% LaTeX file for resume 
% This file uses the resume document class (res.cls)

\documentclass[margin]{res}
\usepackage [brazil]{babel}     % nomes e hifenaçã em português
 
\usepackage{t1enc}              % Permite digitar os acentos de forma normal
\usepackage[utf8]{inputenc} 

\topmargin=-0.5in  % start text higher on the page
\setlength{\textheight}{10in} % increase text height to fit resume on 1 page
\begin{document}  
\name{\textit{Paulo Leonardo Benatto}}

\address{London, UK \\ benatto@gmail.com \\ Phone: 07405110040 \\ Post Code: SW16 2BU }
                           
                        
\begin{resume}                        
 
\section{Summary}       Bachelor in computer science having expertise in software development, good professional relationship and focused at work.
                        With 6 years of professional experience in software development using basically, C/C++, I am now looking for new challenges
                        and opportunities that allow me to learn new technologies and work with new people. I have interest in software development
                        (C/C++, Python, Java, Lua, ShellScript and Go) and Linux Operating System administration.
 
\section{Education}	Universidade Estadual do Oeste do Parana, BSc in Computer Science, December 2007.
  
\section{Experience}

\vspace{-0.1in}
   \begin{tabbing}
   \hspace{2.3in}\= \hspace{1.7in}\= \kill % set up two tab positions
    \textbf{DBA}    \>\>\textbf{Dec 2013 - Feb 2014}\\
    \textit{System Analyst}\\        
    \textbf{Main Technologies}: Linux, Python, Raspberry PI and C;
   \end{tabbing}\vspace{-20pt}      % suppress blank line after tabbing
    \vspace{2mm}
        Member of a team responsible to design and develop a system to analyze vehicle traffic on Brazilian highways.
        Basically my task was read files with a lot of records and parse all information, store in memory and report this
        information in a human readable.  
        In parallel I attended of a project to develop a parking meter system, using python language and Raspberry PI.
        My main task was do inter-process communication using DBus.
\vspace{-0.1in}
   \begin{tabbing}
   \hspace{2.3in}\= \hspace{1.7in}\= \kill % set up two tab positions
    \textbf{SEC+}    \>\>\textbf{Dec 2012 - Dec 2013}\\
    \textit{System Analyst}\\        
    \textbf{Main Technologies}: Linux, Python, Django Framework, JavaScript and C;
   \end{tabbing}\vspace{-20pt}      % suppress blank line after tabbing
    \vspace{2mm}
     Back-end development of web system for intelligent monitoring and management of natural disasters
     using Python and the Django framework. Front-end with Javascript (JQuery, Bootstrap, Google Maps API),
     JSON, HTML5, CSS. Modeling and use of object-relational database and geographic objects with
     PostgreSQL/PostGIS.

   \begin{tabbing}
   \hspace{2.3in}\= \hspace{1.7in}\= \kill % set up two tab positions
    \textbf{Digitro Technology - NDS}    \>\>\textbf{Jan 2012 - Dec 2012}\\
    \textit{System analyst}\\   
    \textbf{Main Technologies}: Linux, C/C++, Java, ShellScript;
   \end{tabbing}\vspace{-20pt}      % suppress blank line after tabbing
    \vspace{2mm}
    Member of the team responsible to design and develop the roadmap features of a product called
    Guardião used to interecept calls and internet trafic of people whose are being investigated by the police. 
    Specifically allocated on the design and development of a module which main goal is to intercept the actions 
    done on Facebook, WhatsApp and Hotmail of a target, using the gathered data to cross reference with another
    possible targets.


   \begin{tabbing}
   \hspace{2.3in}\= \hspace{1.5in}\= \kill % set up two tab positions
    \textbf{Digitro Technology - STE}    \>\>\textbf{Set 2008 - Dec 2011}\\
    \textit{System analyst}\\   
    \textbf{Main Technologies}: Linux, VoIP, C/C++, shellscript, protocols: UDP, TCP, SIP;
   \end{tabbing}\vspace{-20pt}      % suppress blank line after tabbing
    \vspace{2mm}
    
    VoIP systems development basically using C/C++ languages and libraries such as GLib and Sofia-SIP. Working
    in a team I attended big projects, like: \textbf{Softphone}, software program for making telephone calls over
    the Internet using a general purpose computer, rather than using dedicated hardware, and an \textbf{IP Phone}.
    
    Linux administration, python, lua, shell scripting, network protocol analysis are common activities in 
    large projects.
    

\section{Skills Base}  \textit{Operating System}:  Linux (Debian, Ubuntu, CentOS), Windows NT/XP/Vista/7 and OSX;

			\textit{Networkings}: TCP/IP protocol suite;
  
			\textit{Progamming Languages}: C/C++, Pascal, Python, JavaScript, plus some experience with Lua and Java;
  
			\textit{Virtualization}: VirtualBox, VMWare, plus some experience with Xen;

			\textit{Languages}: Fluent in Portuguese, Intermediate in English and Spanish;
 
\section{Open Source Projects}
		\begin{itemize}
		    \vspace{2mm}
		    \item \textbf{libpenetra}: The libpenetra was created with the goal of studying the windows binary format 
		                               known as Portable Executable (PE). With libpenetra you can access all information
		                               about PE binaries. (\texttt{https://github.com/patito/libpenetra}) \vspace{1mm}
		                               
		    \item \textbf{libmalelf}: The libmalelf is an evil library that SHOULD be used for good! It was developed
		                              with the intent to assist in the process of infecting binaries and provide a safe 
		                              way to analyze malwares. (\texttt{https://github.com/SecPlus/libmalelf})\vspace{1mm}
		                              
		    \item \textbf{malelf}: Malelf is a tool that uses libmalelf to dissect and infect ELF binaries. 
		                           (\texttt{https://github.com/SecPlus/malelf})
		\end{itemize}
 
\section{More Info}
    \begin{itemize}
        \item \textbf{Linkedin}: http://www.linkedin.com/in/benatto
         \item \textbf{Github}: https://github.com/patito
    \end{itemize}


\end{resume} 
\end{document}









